\documentclass[a4paper,12pt]{article}%
\usepackage{times}
%\usepackage{cite,citesort}
\usepackage{ifthen,calc}
\usepackage{graphicx,latexsym,color,amsmath,amsfonts,amssymb,bm,eucal,mathrsfs}
\usepackage{amsmath}
\usepackage{amsfonts}
\usepackage{amssymb}
\usepackage{graphicx}
\usepackage{bm}
\usepackage{amsthm}
\usepackage{authblk}
\usepackage{textcomp}%
\setcounter{MaxMatrixCols}{30}
%TCIDATA{OutputFilter=latex2.dll}
%TCIDATA{Version=4.10.0.2359}
%TCIDATA{LastRevised=Tuesday, April 01, 2014 09:23:42}
%TCIDATA{<META NAME="GraphicsSave" CONTENT="32">}
\newtheorem{com}{Comment}
\newtheorem{com1}{Comment 1}
\newtheorem{com2}{Comment 2}
\newtheorem{com3}{Comment 3}
%\input{mymathwrite}
\newtheorem{theorem}{Theorem}
%\include{figuretable}
%\include{mathinclude}
\begin{document}
\date{} \title{Replies to comments on\\Manuscript
  ENGI--D--13--00133\\"Fourier Methods for Harmonic Scalar Waves in
  General Waveguides"} \author{Anders Andersson, B\"orje Nilsson, and
  Thomas Biro}
\maketitle

We would like to thank the two reviewers for many valuable comments
that have helped us improving the paper. Based on these comments, the
paper is completely rewritten. Special attention has been paid to
motivating the method and the presentation and the discussion of the
results. We agree with the opinion that the motivations for the method
were absent and that the presentation of the results was poor in the
previous version.
\section*{Referee 1}


{\it Referee's comment}\\
The presentation of the math problem is poor. Eqs (2) and (3) seem to
be an eigenvalue problem. But the author actually solved a boundary
value problem. In the numerical example, the definition of boundary
conditions is misleading. A source at left is mentioned in page 15,
while in section 5.1, boundary conditions are defined on elements of
conformal mappings. Please unify the definitions. Moreover, can these
conditions be rewritten in the form of Eq. (3). If not, the original
math problem has to be re-defined.\\ 
{\it Authors' comments and actions}\\
The former implicit definition of the source as an incident wave is
now made explicit in section 2 making it clear that a scattering
problem is formulated and later solved. Then it should not cause
confusion in keeping the use of the conformal mapping in section 5.1
to define the intervals where the non-zero admittance is applied.\\ 
\newline
{\it Referee's comment}\\
Many different methods are mentioned in Introduction section. Please
provide key references and a brief description of these methods.\\ 
{\it Authors' action}\\
The description of different methods in the introduction is expanded
and key references are added.


\section*{Referee 2}

IMPORTANT POINTS: \textit{in italics text from the manuscript;}\\
{\it Referee's comment}\\
(1): Introduction: \textit{The problem appears in many applications,
  in acoustics, optics, electrodynamics and quantum physics.} The
authors should provide one reference for each
theme mentioned. I would add water waves and the reference:\\
Shi, A., Teng, M.H. \& Wu, T.Y., Propagation of solitary waves through
significantly
curved shallow water channels. J. Fluid. Mech., 362 (1998) 157176.\\
Note this is nonlinear, fully time dependent problem.\\
{\it Authors' action}\\
References for the themes are mentioned and the new theme water waves
is added together with the suggested reference.\\
\newline
{\it Referee's comment}\\
(2): Preliminaries: \textit{The purpose of this article is exactly
  this, to investigate and to show how Fourier methods can be used to
  solve wave scattering problems in a waveguide with geometry and
  boundary conditions that exceed the ordinary school book examples.}
I believe there has to be a stronger motivation for this article. Note
at the end, by the time we reach the Conclusions we have: We can only
conclude that the combination of semi-analytic techniques evolved here
is a well working alternative to different FEM- based numerical
methods. As an interested reader I felt frustrated of not having any
idea of which method is better: this or the FEM? Is the present one
faster? Is there any comparison in terms of performance regarding
computer time? If not, then why is this an ALTERNATIVE?! What is the
benefit if a computer has to be used anyway with
the FEM or this one?\\
{\it Authors' comment}\\
We agree that the motivations for the method was absent in the
previous version of the paper.\\
{\it Authors' actions}\\
The introduction has been rewritten in order to present motivations of
non-mathematical nature for the method. It is based on industrial
requirements on wave simulation tools originating partially from the
second author's previous industrial experience. In brief, more than
one type of simulation tools is required.

From these motivations the revised purpose for the paper is
\emph{\ldots to demonstrate that semi-analytical methods based on
  Fourier Analysis can solve time harmonic scattering problems in
  waveguides with complicated geometry and general normal impedance or
  admittance boundary conditions.}

An extended summary of the paper is presented including a balanced
conclusion. The importance of improved physical understanding is
stressed. The revised conclusion and summary is: \emph{ In summary, we
  conclude, based primarily on requirements from industry, that more
  than one type of time harmonic waveguide simulation tool is
  required. It is demonstrated that Fourier methods based on Fourier
  Analysis provides one such tool. Its accuracy is checked against FE
  Simulations for a general two-dimensional waveguide with normal
  admittance boundary conditions at low and medium frequencies. For
  the current investigation with non-zero normal boundary admittance,
  the Fourier method, with its present implementation, is considerably
  slower than the FE method that is more memory demanding. However,
  for inverse engineering involving tuning of straight waveguides, the
  Fourier method is an attractive alternative including time
  aspects. The prime motivation for the Fourier method is its added
  physical understanding primarily
  at low frequencies.}\\
\newline
{\it Referee's comment}\\
(3): Page 5: \textit{When performing these calculations, interactions
  from the two ends of the block must be avoided, and hence, each
  block is assumed to be an infinitely long waveguide (...).} How was
this infinitely long dealt with in the simulations. Since one the
authors has experience with the Schwarz-Christoffel Toolbox by
Driscoll [1] this could have been tested and displayed here. One does
not need to go too far, correct? How
far?\\
{\it Authors' comment}\\
This text was intended to motivate why the BB method requires that
both ends of each block is an infinitely long waveguide corresponding
to anechoic terminations. The text was not intended to connect to the
Schwarz-Christoffel Mapping.\\
{\it Authors' action}\\
The following text is added for clarification: \emph{It is assumed
  that the scattering matrix for each block is determined for anechoic
  terminations. Otherwise, the interactions due to the termination
  should have been included in the Building Block Method making it
  much more complicated. To this end each block is assumed to be an
  infinitely long waveguide with parallel straight walls and constant
  boundary conditions outside some bounded transition region, when
  determining its scattering properties; in applications the straight
  waveguide parts may be of shorther or even vanishing length.}\\
In the end of Section 5.2, it is now clearly stated which intervals
are used for the two blocks in the model example.\\
\newline
{\it Referee's comment}\\
(4): Page 8: \textit{Both methods are built on the Schwarz-Christoffel
  mapping, (...), mean- ing that no singularities are introduced by
  the mapping.} This is along the lines of the
item above. Note that in\\
Nachbin, A. \& Sim\~oes, V.S. 2012 Solitary Waves in Open Channels
with Abrupt
Turns and Branching Points. J. Nonlin. Math. Phys., Vol. 19, Suppl. 1, 1240011,\\
the same strategy was used for the singularities of the conformal
mapping. Also note that the reflection-transmission problem at sharp
turns was done in a very straightforward
and accurate fashion for a nonlinear evolution problem, namely more complex.\\
{\it Authors' action}\\
The suggested reference is added.\\
\newline
{\it Referee's comment}\\
(5): Page 11: \textit{It is worth noticing that the Riccati equations
  (...) can for a countable set of k contain singularities (...), for
  details and examples see for example [6].} The authors should
mention in words what is this singularity problem. And why they resist
a numerical solution .\\
{\it Authors' comment and action}\\
The wording "resist a numerical solution" was unfortunate. We have
clarified the situation also mentioning that the stiffness was
tractable with standard methods for our example.\\
\newline
{\it Referee's comment}\\
(6): Page 16: Figure 6 is very bad. I can not read the numbers on
either axis, nor know exactly what-is-what?! How many $\Phi_j$ are
there? What do I conclude from this figure?\\
{\it Authors' action}\\
By using better type sizes on axes and in legends, many figures in
Section 5.4 have been made more readable. The caption in Figure 6 now
includes further relevant information and additional comments on the
figure are included in the text.\\
\newline
{\it Referee's comment}\\
(7): Page 16-17: Why $k$ = 15 was chosen, was a significant mode? As
mentioned in the Conclusions why is this a low frequency mode? Why is
figure 7 a good display of the results, or of the quality of the
method? Why in the top of table 1 three $\Phi$'s are
considered while at the bottom only 2? Why figure 8 has only $\Phi_0$?\\
{\it Authors' comment and action}\\
The model problem has been solved for frequencies in the interval
$0<k\le20$. All statements, indicating the upper part of this interval
as beeing ``low frequencies'' have been removed. Instead, we use
terms like  ``low and medium'' frequencies.\\
The former Figure 7 is removed from the article.\\
Table 1 and its caption is slightly changed to increase its
understandability. \\
\newline
{\it Referee's comment}\\
(8): Page 18: \textit{Finally, it is possible to extend the techniques
  to cover three-dimensional problems(...)} Nowhere 3D is done so it
should not be mentioned at the start at page 3 and beyond. Also the
doubly connected region in Figure 1 is misleading since holes are
NOT considered. This should be removed.\\
{\it Authors' comment and action}\\
Three-dimensional problems are not any longer included in the initial
formulation of the problem. However, a note on the possibilty to
extend the methods to cover 3D is still present in the discussion
section. A reference to an article which discusses the matter is
included. \\
\newline
{\it Referee's comment}\\
(9): Conclusion: At this point as a reader I am not sure I want to use
the method proposed, which might be useful. It seems time
consuming. Is it? It has restrictions.  So we go back to item 1 above
and read \textit{The purpose of this article is exactly this, to
  investigate and to show how Fourier methods can be used to solve
  wave scattering problems in a waveguide with geometry and boundary
  conditions that exceed the ordinary
  school book examples.}\\
\newline I would like to be convinced that a reader would want to use
this method and why.
Or else the paper should not be published.\\
{\it Authors' action}\\
See comments on point (2).\\
\newline
\textbf{Minor points:}\\
{\it Referee's comment}\\
(a): typo/Abstract: equtions\\
{\it Authors' action}\\
The abstract is rewritten.\\
\newline
{\it Referee's comment}\\
(b) Intro: This sentence is quite long and confusing. \textit{For more
  complex geome- tries or boundary conditions, purely numerical
  methods or more precisely, finite element methods, have often during
  recent years, due to evolution of both the methods and the
  computers, become the natural choice for solving Helmholtz equation.}\\
{\it Authors' action}\\
The sentence is reformulated.\\
\newline
{\it Referee's comment}\\
(c): \textit{We can only conclude that the combination of
  semi-analytic techniques} \textbf{evolved}
\textit{here is a well working alternative to different FEM-based numerical methods.}\\
{\it Authors' action}\\
The sentence is no longer present in the article.


\end{document}
